\chapter{BEL Commons: an environment for exploration and analysis of networks encoded in Biological Expression Language}\label{ch:belcommons}

\section*{Preface}

The following publication describes BEL Commons: a web application that enables exploration and analysis of networks encoded in \ac{BEL} as well as the integration of several disparate features from other prominent web applications for systems and networks biology.
It highlights the way the web application allows for integrative exploration of several publicly available knowledge bases and provides an example of data-driven analysis using publicly available differential gene expression data sets related to \ac{AD}.

\vspace*{\fill}

\includepdf[pages={-}]{articles/belcommons.pdf}

\section*{Postface}

One of the issues of the accessibility of \ac{BEL} mentioned in Chapter~\ref{ch:pybel} was the ability for users to use previously published algorithms.
With a web interface that guides users through applying those algorithms, BEL Commons moves the community closer to overcoming that issue.
More generally, BEL Commons enables users who are less familiar with programming to access the \ac{BEL} ecosystem.
Finally, the article motivated several improvements, such as the deeper integration of the PyBEL ecosystem with \ac{INDRA}~\cite{Gyori2017} in order to support semi-automated curation and enrichment of knowledge graphs as well as the importation of other formats like \ac{BioPAX} and \ac{SBML} that are realized in Chapter~\ref{ch:recuration}.
