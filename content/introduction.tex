\chapter{Introduction}

\section{Nomenclature}

\subsection{Issues with Gene Nomenclature}

The nomenclature of genes and gene products is a particularly egregious example of nomenclature within the biomedical domain.
Genes often have several names as well as several incomprehensible acronyms, or gene symbols.
For example, the \ac{HGNC}~\cite{Yates2017} and Entrez Gene~\cite{Maglott2011} database list that the human gene, microtubule associated protein tau (hgnc:HGNC:6893, ncbigene:4137), has previously been named the G protein
$\beta1$/$\gamma2$ subunit-interacting factor 1 and the protein phosphatase 1,
regulatory subunit 103.
Like with many genes, it is often acronymized to MAPT in text, but it has additionally been previously referenced with DDPAC, FLJ31424, FTDP-17, MAPTL, MGC13854, MTBT1, MTBT2, MSTD, PPND, and PPP1R103.

Neither genes' names nor their gene symbols convey their host species, which leads to further ambiguities in articles discussing orthologs in model organisms.
The organization responsible for mouse gene nomenclature, \ac{MGI}~\cite{Blake2017}, names the mouse orthologous gene as microtubule-associated protein tau (mgi:MGI:97180, ncbigene:17762) and lists the gene symbol as Mapt.
In this example, the name varies from the human ortholog with the introduction of a dash between "microtubule" and "associated."
The gene symbol differs only in capitalization.
Similarly, the organization for rat genome nomenclature, \ac{RGD}~\cite{Shimoyama2015}, names the rat orthologous gene as microtubule-associated protein tau (rgd:69329; ncbigene:29477) - exactly as in \ac{MGI}.
While these orthologs from common model organisms have had related names, organisms with genetic drift such as Zebrafish have several orthologs named microtubule-associated protein tau a (zfin:ZDB-GENE-081027-1) and microtubule-associated protein tau b (zfin:ZDB-GENE-081027-2) whose gene symbols are listed as mapta and maptb, respectively.
Other orthologs to human microtubule-associated protein tau can be found in Homologene (homologene:74962), Ensembl~\cite{Zerbino2018}, \ac{HGNC}, \ac{MGI}, PomBase, \ac{RGD}, Xenbase, and \ac{ZFIN}.
The \ac{HCOP}~\cite{Wright2005} aggregates these and several other sources of curated and predicted orthologies.

\subsection{Nomenclature Consortia of Genes and Proteins}

Most biologically relevant named entities go by many names.
For example, many genes were discovered and characterized in different labs and therefore named differently.
As resources for exchanging genomic and protein sequences have become more ubiquitous in the last thirty years, it has become easier to reduce those duplicates.
However, this does not solve the problem of establishing a canonical name for each entity.
As alluded to in the previous section, several committees and consortia have formed to standardize the nomenclature used for genes for each species (Table~\ref{table:gene_nomenclature_databases}).

\begin{table}
    \centering
    \begin{tabular}{ c c c }
        Organism & Database & Reference \\
        \hline
        Human & HGNC &\cite{Yates2017} \\
        Vertebrae & VGNC &\cite{Yates2017} \\
        Mouse & MGI &\cite{Blake2017} \\
        Rat & RGD &\cite{Shimoyama2015} \\
        Zebrafish & ZFIN &\cite{Howe2013}  \\
        Drosophila (fly) & FlyBase &\cite{Thurmond2019}\\
        Xenopus (frog) & Xenbase &\cite{Karimi2018}  \\
        Yeast & SGD &\cite{Cherry2012} \\
        - & Entrez Gene &\cite{Maglott2011}  \\
    \end{tabular}
    \caption{Example model organism gene nomenclature databases}
    \label{table:gene_nomenclature_databases}
\end{table}

\subsection{Nomenclature Consortia of Other Entities}

Besides gene nomenclature, there are several other biologically relevant physical entities and higher-order processes that have have the same issues in nomenclature.
Further, for higher-order processes like pathways, mechanisms, and biological processes, it not only remains unclear what to name each, but where their boundaries lie.
However, deference to entities beyond genes and proteins are required to fully describe complex biology.
Thus, several groups have attempted to standardize and control their nomenclature (Table~\ref{table:other_nomenclature_databases}).

\begin{table}
    \centering
    \begin{tabular}{ c c }
        Entity Type & Resources \\
        \hline
        Transcripts & Ensembl, miRBase \\
        Proteins & UniProt \\
        Protein Families & InterPro, neXtProt, FamPlex, ExPASy, Signor \\
        Protein Complexes & Complex Portal, FamPlex, Signor, Gene Ontology \\
        Biological Processes & Gene Ontology MeSH \\
        Pathways & Reactome, WikiPathways, KEGG \\
        Conditions and Phenotypes & Disease Ontology, Human Phenotype Ontology, MeSH
    \end{tabular}
    \caption{Example entity types of interest in the biomedical domain and corresponding nomenclature sources}
    \label{table:other_nomenclature_databases}
\end{table}

\subsection{Practical Considerations in Named Entity Recognition}

Besides further synonyms and morphological variations, Bachman et al. (2018)~\cite{Bachman2018} outlined several affixes corresponding to post-translational modification state, experimental context, or other categories (Table~\ref{table:affix_categories}).

\begin{table}
    \centering
    \begin{tabular}{ c c }
        Affix Category & Example \\
        \hline
        Experimental context & eGFP-{Gene name} \\
        Protein state & phospho-{Gene name} \\
        Inhibitor & shRNA-{Gene name} \\
        Generic descriptor & Proto-oncogene {Gene name} \\
        Species & mmu-{Gene name} \\
        mRNA grounding & {Gene name} mRNA
    \end{tabular}
    \caption{Examples of affix categories in the FamPlex ontology adapted from Table 2 of~\cite{Bachman2018}}
    \label{table:affix_categories}
\end{table}

Bachman \textit{et al.} also considered issues with recognizing proteins, protein families, and protein complexes~\cite{Bachman2018}.
The biomedical literature often references multi-protein families (e.g., RAS, AKT) and multi-subunit complexes (e.g., NF-kB, AP-1) rather than their constituent proteins.
For example, the protein family of phospholipase C enzymes, more commonly referenced as PLC, contains not only individual genes (e.g., PLCE1), but also subfamilies such as PLCG, which contains the genes PLCG1 and PLCG2.
The NF-kB complex comprises five proteins (i.e., RELA, RELB, REL, NFKB1, and NFKB2) and poses the further challenge of how named entities should be interpreted after recognition.

\subsection{Automating Named Entity Recognition}

In practice, automating the process of recognizing named entities has three major tasks: coreference resolution, named entity recognition, and entity linking.

\subsubsection*{Coreference Resolution}

During the process of coreference resolution, antecedents or anaphors are identified and connected to their preceding or succeeding words or phrases.
In the example sentences, "The EGFR belongs to a family of protein-tyrosine kinase receptors.
It is activated by the binding of EFG.", it identifies that the subject of the second sentence, it, refers to EGFR.
There are several classical rule-based coreference resolution algorithms including the syntax-based Hobbs theory~\cite{Hobbs1978}, discourse-based centering theory~\cite{Brennan1987}, and syntactic knowledge-based RAP algorithm~\cite{Brennan1987}.
However, recent improvements to coreference resolution have focused on four categories of machine learning techniques: mention-pair models~\cite{Soon2001,Ng2002,Bengtson2008}, entity-mention models~\cite{Luo2004,Yang2004,Yang2008}, mention-ranking models~\cite{Lee2011,Denis2007,Rahman2009,martschat2015}, and cluster-ranking models~\cite{Rahman2011,Ma2014,Clark2016}.
Recent work has focused on using recurrent neural networks with architectures such as the bi-directional long-short-term memory~\cite{Li2018} and variants such as the bi-directional long-short-term memory conditional random field~\cite{Giorgi526244}.

\subsubsection*{Named Entity Recognition}

During the process of named entity recognition, words or phrases are identified to have domain-specific, non-trivial meaning.
In the biomedical domain, named entity recognition is used to identify proteins~\cite{Hsu2008,Leaman2008, Hakenberg2011,Wei2015}, chemicals~\cite{Leaman2015,Corbett2018,Giorgi526244}, diseases~\cite{Leaman2013,Giorgi526244}, taxa~\cite{Gerner2010,Wei2012}, and other entity types listed in Table~\ref{table:other_nomenclature_databases}.

Unlike the clear difference between the rule-based and natural language processing models of coreference resolution versus the machine learning techniques, named entity recognition workflows often contain a mixture of preprocessing steps, rule-based feature generation, statistical models, machine learning models, and postprocessing steps.
For example, Lee \textit{et al.}~\cite{Lee2015} defined six feature classes for training a disease recognition model using a conditional random field.
First, they generated morphological features that contained the original tokens, the corresponding stemmed tokens, and their affixes.
Second, they generated features based on their terminology of trigger words related to diseases, body parts, and human ability.
Third, they generated classical part-of-speech features.
Fourth, they transformed the original tokens to remove continued vowels, such as the superfluous "u" in the british spelling of tumour.
Fifth, they generated features using a dictionary lookup on the MEDIC~\cite{Davis2012} disease dictionary.
Sixth, they annotated abbreviations using BIOADI~\cite{Kuo2009}.

Other workflows have opted to use end-to-end machine learning since the recent advent of word embedding techniques such as word2vec~\cite{Mikolov2013} and GloVeZ\cite{Pennington2014} and deep learning techniques like the long-short-term memory conditional random field architecture~\cite{Lample2016}.
Both variants require carefully constructed and annotated corpora such as GENIA~\cite{Kim2003} or those provided by the various BioCreative challenges (https://biocreative.bioinformatics.udel.edu).
With machine learning techniques, the annotations become more important as they are also necessary for supervised learning.


\subsubsection*{Entity Linking}

During the process of entity linking (i.e., normalization, grounding), named entities that have been recognized in the previous step are matched to terms in controlled vocabularies or databases.
For example, this means that the token MEK1 should be recognized as a synonym of the MAP2K1 gene and subsequently grounded with the HGNC identifier hgnc:HGNC:6840, Entrez Gene identifier ncbigene:5604, UniProt identifier uniprot:Q02750, and any other desired equivalent identifiers.
Note that each of these identifiers is written using the \ac{CURIE} style (e.g., an identifier prefixed with a namespace) and that the HGNC identifier includes a redundant mention of the namespace within the identifier.
This is discussed in the next section.
Practically, mappings between equivalent terms in databases are manually curated by database maintainers and data stewards.
As more of the nomenclatures, terminologies, and taxonomies useful to the bioinformatics community move towards ontological formats like OWL and OBO, these mappings become more reusable.
Following, tools like the \ac{EBI} \ac{OLS}; Cote et al., 2006)~\cite{Cote2006} and emerging \ac{EBI}  Ontology Xref Service (https://www.ebi.ac.uk/spot/oxo) have been able to prove the community with technical solutions for storing, indexing, and looking up information about these terms.
