\chapter{Re-curation and rational enrichment of knowledge graphs in Biological Expression Language}
\label{ch:recuration}

\section*{Preface}

Following Chapters~\ref{ch:pybel} and~\ref{ch:belcommons} that laid the groundwork for handling and exploring knowledge graphs encoded in \ac{BEL}, the following paper describes the development and application of two workflows for 1) ensuring the quality of knowledge graphs encoded in \ac{BEL} and 2) enriching these knowledge graphs with semi-automated curation that leverages large-scale information extraction and natural language processing systems.
It presents an evaluation and comparison to previous semi-automated curation workflows using the metrics for curation overhead and efficiency described by Rodriguez-Esteban~\cite{Rodriguez-Esteban2015}.

\vspace*{\fill}

\includepdf[pages={-}]{articles/recuration.pdf}

\section*{Postface}

Continuing with the goals of developing the ecosystem around \ac{BEL} and PyBEL, the workflows and all resulting manually curated results from this paper have been made freely and openly available to the community.
The first workflow for quality control of \ac{BEL} documents as they have been curated and integration with version control systems suggests more sustainable curation practices and enables the already difficult job to be managed slightly easier.
The second workflow allows for the usage of massively extracted content from unstructured text and automated enrichment of knowledge graphs, which further reduces this burden.
