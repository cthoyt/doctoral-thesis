\chapter{A systematic approach for identifying shared mechanisms in epilepsy and its comorbidities}
\label{ch:epicom}

\section*{Preface}

The epidemiological evidence of comorbidity between \ac{AD} and epilepsy motivated further investigation of their potential shared molecular mechanisms of pathogenesis.
These shared mechanisms present opportunities for transfering knowledge in one domain to another and lead to obvious drug repositioning mechanisms for compounds tested for one indication that affects a pathway shared by another indication.

The final publication in this thesis describes the curation of a disease-specific knowledge assembly for epilepsy, its categorization into signatures in NeuroMMSig~\cite{Domingo-Fernandez2017}, and the development and application of the comparative mechanism enrichment workflow to identify shared molecular mechanisms with \ac{AD}.
It presents an application scenario in which a knowledge-driven approach was used to hypothesize shared pathways between epilepsy and \ac{AD} that might be affected by the drug, carbamazepine, which has been observed through epidemiological studies to have positive therapeutic benefits in both disease contexts.

\vspace*{\fill}

Reprinted with permission from "Hoyt, C. T., \textit{et al.} (2018). A systematic approach for identifying shared mechanisms in epilepsy and its comorbidities. \textit{Database}, 2018(1)".
Copyright © Hoyt, C.T., \textit{et al.}, 2018.

\includepdf[pages={-}]{articles/epicom.pdf}

\section*{Postface}

The analysis presented in this publication ranked disease-specific mechanisms in \ac{AD} and epilepsy that are likely targeted by carbamazepine and ultimately lead to the hypothesis that the GABA-ergic receptor pathway was central to its multi-indication effect.
Importantly, this investigation was advantageous over black-box machine learning models because the underlying knowledge assemblies are self-explanatory and based on publications published in molecular biology and epidemiology.
After, the prospects of applying these techniques in a more automated fashion in order to investigate many more drugs and disease combinations were discussed.

In the spirit of reproducible and reusable science, the instructions, scripts, and the Epilepsy Knowledge Assembly have been made publicly available at \url{https://github.com/neurommsig-epilepsy} in order to enable other scientists to reproduce ours and conduct their own investigations.

Given the automation enabled by the enrichment methods described in Chapters~\ref{ch:recuration} and~\ref{ch:bio2bel}, this analysis can be more heavily automated to provide new insights as the underlying knowledge assemblies grow and increase in granularity.
As several new neurodegenerative diseases (e.g., multiple sclerosis, amyotrophic lateral sclerosis, Huntington's disease) are currently being added to NeuroMMSig, new shared mechanisms will become apparent and lead to new hypotheses to be tested in epidemiological studies and later for drug repositioning.
