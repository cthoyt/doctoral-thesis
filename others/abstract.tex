\chapter*{Abstract}

This thesis begins with the acknowledgement of the acceleration of generation of knowledge within the biomedical domain.

The first two papers (PyBEL and BEL Commons) build an ecosystem for handling this knowledge during curation, application of algorithmics, and visualization.
The second two papers revolve around enabling the acquisition of high-granularity knowledge from structured sources on a massive scale (Bio2BEL) and supporting the semi-automated curation of new content at high speed and precision (Re-curation and Rational Enrichment).

Finally, after building the ecosystem and acquiring the content, the third part of this thesis revolves around the applications of biological knowledge graphs in simulation and modeling.
This includes agent-based modeling using biological knowledge graphs as priors (BEL2ABM), the application of network representation learning to prioritize nodes in biological knowledge graphs based on corresponding experimental measurements (GuiltyTargets), and finally, the use of biological knowledge graphs and development of algorithmics to deconvolute the mechanism of action of drugs, that could also serve as a drug repositioning candidate identifier (EpiCom).

Ultimately, the this thesis lays the groundwork for production-level applications of drug repositioning algorithms and other knowledge-driven approaches to analyzing biomedical experiments.
