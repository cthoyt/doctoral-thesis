\chapter*{Abstract}

This thesis has the two goals to:

\begin{enumerate}
    \item Improve methods for curation and semantic data integration to generate high granularity biological knowledge graphs
    \item Develop novel methods for using prior biological knowledge to propose new biological hypotheses
\end{enumerate}


The first two publications about PyBEL and BEL Commons build an ecosystem for handling biological knowledge graphs encoded in the Biological Expression Language (BEL) throughout the stages of curation, visualization, and analysis.
The second two publications enable the acquisition of high-granularity knowledge with low contextual specificity from structured biological data sources on a massive scale (Bio2BEL) and support the semi-automated curation of new content at high speed and precision (Re-curation and Rational Enrichment).

After building the ecosystem and acquiring content, the final three publications in this thesis demonstrate three different applications of biological knowledge graphs in modeling and simulation.
They include agent-based modeling for simulation of neurodegenerative disease biomarker trajectories using biological knowledge graphs as priors (BEL2ABM), the application of network representation learning to prioritize nodes in biological knowledge graphs based on corresponding experimental measurements to identify novel targets (GuiltyTargets), and finally, the use of biological knowledge graphs and development of algorithmics to deconvolute the mechanism of action of drugs, that could also serve as a drug repositioning candidate identifier (EpiCom).

Ultimately, the this thesis lays the groundwork for production-level applications of drug repositioning algorithms and other knowledge-driven approaches to analyzing biomedical experiments.
