\chapter{Introduction}

\section{Nomenclature}

\subsection{Issues with Gene Nomenclature}

The nomenclature of genes and gene products is a particularly egregious example of nomenclature within the biomedical domain. Genes often have several names as well as several incomprehensible acronyms, or gene symbols. For example, HGNC and Entrez Gene list that the human gene, microtubule associated protein tau (hgnc:HGNC:6893, ncbigene:4137), has previously been named the G protein $\beta1$/$\gamma2$ subunit-interacting factor 1 and the protein phosphatase 1, regulatory subunit 103. Like with many genes, it is often acronymized to MAPT in text, but it has additionally been previously referenced with DDPAC, FLJ31424, FTDP-17, MAPTL, MGC13854, MTBT1, MTBT2, MSTD, PPND, and PPP1R103.

Neither genes' names nor their gene symbols convey their host species, which leads to further ambiguities in articles discussing orthologs in model organisms. The organization responsible for mouse gene nomenclature, Mouse Genome Informatics (MGI), names the mouse orthologous gene as microtubule-associated protein tau (mgi:MGI:97180, ncbigene:17762) and lists the gene symbol as Mapt. In this example, the name varies from the human ortholog with the introduction of a dash between "microtubule" and "associated." The gene symbol differs only in capitalization. Similarly, the organization for rat genome nomenclature, the Rat Genome Database (RGD), names the rat orthologous gene as microtubule-associated protein tau (rgd:69329; ncbigene:29477) - exactly as in MGI. While these orthologs from common model organisms have had related names, organisms with genetic drift such as Zebrafish have several orthologs named microtubule-associated protein tau a (zfin:ZDB-GENE-081027-1) and microtubule-associated protein tau b (zfin:ZDB-GENE-081027-2) whose gene symbols are listed as mapta and maptb, respectively. Other orthologs to human microtubule-associated protein tau can be found in Homologene (homologene:74962), Ensembl (Zerbino et al., 2018), HGNC, MGI, PomBase, RGD, Xenbase, and ZFIN. The HGNC Comparison of Orthology Predictions (HCOP; Wright et al., 2005) aggregates these and several other sources of curated and predicted orthologies.

- Daniel R. Zerbino, et al. (2018) Ensembl 2018. PubMed PMID: 29155950.

- Wright MW, Eyre TA, Lush MJ, Povey S and Bruford EA. HCOP: The HGNC Comparison of Orthology Predictions Search Tool. Mamm Genome. 2005 Nov; 16(11):827-828. PMID:16284797

\subsection{Issues with Gene Nomenclature}

Most biologically relevant named entities go by many names. For example, many genes were discovered and characterized in different labs and therefore named differently. As resources for exchanging genomic and protein sequences have become more ubiquitous in the last thirty years, it has become easier to reduce those duplicates. However, this does not solve the problem of establishing a canonical name for each entity. As alluded to in the previous section, several committees and consortia have formed to standardize the nomenclature used for genes for each species (\ref{table:gene_nomenclature_databases}).

\begin{table}
\centering
\begin{tabular}{ c c c }
  Organism & Database & Reference \\
	\hline
 cell4 & cell5 & cell6 \\
 cell7 & cell8 & cell9
\end{tabular}
\caption{A non-exhaustive list of model organism gene nomenclature databases}
\label{table:gene_nomenclature_databases}
\end{table}
